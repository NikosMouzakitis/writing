\documentclass{article}
\usepackage[utf8]{inputenc}
\usepackage{booktabs} % For professional table formatting
\usepackage{array}

\begin{document}

\section*{Bibliographical references}

Frolov et al. (2020) proposed a novel approach to analyze multiplex brain networks using recurrence quantification analysis (RQA) 
on EEG data. The authors demonstrated how recurrence-based synchronization indices can effectively capture 
both within-frequency (intralayer) and cross-frequency (interlayer) functional connectivity during cognitive tasks. 
Their work showed that RQA is particularly suitable for analyzing non-stationary EEG signals and revealed
important insights about the evolution of functional connectivity patterns during prolonged cognitive tasks.

Kang et al. (2023) \cite{kang}, in their study investigate the periodic dynamics of the 
Default Mode Network (DMN) in schizophrenia using 
RQA and Cross-Recurrence Quantification Analysis (CRQA) on resting-state fMRI data. 
Findings include decreased determinism (DET) between key DMN regions 
(vMPFC-PCC and vMPFC-precuneus) in first-episode schizophrenia patients, 
indicating disrupted predictability of functional interactions. 
The study highlights the potential of RQA/CRQA as tools for capturing nonlinear brain dynamics 
and their utility in distinguishing schizophrenia patients from healthy controls with 77\% 
classification accuracy.

Researchers in \cite{rezaei},
have applied RQA on resting-state fMRI data from TgF344-AD rats in order 
to detect early stage of Alzheimer's disease biomarkers. 
Analysis has been conducted on the Default Mode-Like Network (DMLN) 
using RQA metrics(entropy, recurrence rate, determinism and average diagonal line length) 
and revealed significant changes in regions like 
the basal forebrain (BFB), hippocampal fields (CA1, CA3), and visual 
cortices (V1, V2). 
On the study's findings are included reduced predictability in 
wild-type (WT) rats with aging, while AD rats exhibited less decline
in predictability, suggesting compensatory mechanisms. 
The study highlights RQA's sensitivity to nonlinear dynamics 
in preclinical AD, offering potential for early diagnosis. Also the code of the
research is publicly available.

Lameu et al. \cite{lameu2018} investigated burst phase synchronization in neural networks using 
RQA. The authors employed coupled Rulkov maps to model bursting neurons in both single small-world networks
and clustered network-of-networks architectures. Their spatial RQA approach successfully identified 
synchronized neuron groups and quantified their sizes during synchronization transitions. The study
demonstrated that RQA measures (recurrence rate, laminarity, and structure size) provide complementary 
information to traditional order parameters, particularly for detecting localized synchronization patterns. 
This work is significant for EEG analysis as it shows RQA's capability to detect phase synchronization 
in complex networks - a key feature in functional brain connectivity studies.


Lombardi et al. \cite{Lombardi2014} investigate the nonlinear properties of fMRI BOLD signals 
during a working memory task in schizophrenic patients and healthy controls. 
Using RQA, analysis has been performed on the recurrence plots for the quantification of 
determinism ($D$), trapping time ($TT$), and maximal vertical line length ($V_{\max}$) 
in functionally relevant brain clusters. 
Outcome revealed differences in nonlinear dynamics among the groups, 
and more specific in working memory and default mode network areas. 
This study highlights the potential of RQA for discriminating pathological brain states and understanding 
functional connectivity in complex systems. While their work focused on fMRI, the methodology is adaptable
to EEG, which offers higher temporal resolution for capturing rapid neural dynamics.

Author in \cite{pitsik} (2025), investigated changes related to aging in brain sensorimotor systems using 
RQA and theta-band functional connectivity in EEG signals. 
In the study a VR experimental paradigm was utilized with auditory 
stimuli across different age groups. 
Key findings revealed that elderly subjects showed decreased EEG complexity during motor preparation stages as 
measured by RQA metrics ($\Delta$RR and $\Delta$RTE), 
and had increased theta-band functional connectivity highlighting the potential of RQA in detecting 
age-related biomarkers that were not evident in conventional spectral analysis. 


Guglielmo et al. (2022) \cite{guglielmo} demonstrated the capability of 
RQA features which were extracted by EEG signals for the purpose of classification
of cognitive performance during mental arithmetic tasks. Frontal and parietal EEG signals 
have been analyzed from 36 participants by extracting six RQA metrics (recurrence rate, determinism, 
laminarity, entropy, maximum diagonal line length and average diagonal line length) from four electrodes (F7, Pz, P4, Fp1). 
By further application of machine learning(ML) classifiers such as SVM, Random Forest, and Gradient Boosting, researchers
they reached accuracy of classification above 0.85, showing the potential that RQA hold for generalizing on nonlinear dynamics.

%rewrite kala.
Lopes et al. (2020)\cite{lopes} have proposed a combinatorial framework 
mixing RQA with dynamic functional network (dFN) analysis,
applying it to both MEG and stereo EEG data. 
The methodology they described is split into five steps: data segmentation, 
functional network inference, distance computation alongside networks, 
recurrence plot construction and finally RQA. 
The study demonstrated that functional networks in epilepsy 
patients recur more quickly than in healthy controls, suggesting RQA of 
dFNs could serve as a potential biomarker.
For EEG applications, they showed that pre-ictal networks exhibit higher recurrence rates 
than post-ictal periods, with the $\tau$-recurrence rate ($RR_{\tau}$) proving particularly 
effective for seizure detection.

\begin{table}[h]
\centering
\caption{Comparison among the retrieved studies using recurrence analysis}
\label{tab:comparison}
\begin{tabular}{@{}lcccc@{}}
\toprule
\# & Reference & Modality & Analysis Methods & Network Type \\
\midrule
1 & Frolov et al. (2020) & EEG & RQA, CRQA & Multiplex functional networks \\
2 & Kang el al. (2023) & fMRI & RQA, CRQA & Default mode network \\
3 & Rezaei el al. (2023) & fMRI & RQA & Default model like network \\
4 & Lameu et al. (2018) & --- & RQA & Small-world \& clustered networks \\
5 & Lombardi et al. (2014) & fMRI & RQA & schizophrenia,working memory \\
6 & Pitsik E. (2025) & EEG & RQA & aging \\
7 & Guglielmo et al. (2022) & EEG & RQA & cognitive tasks \\
8 & Lopes et al. (2020) & sEEG, MEG & RQA & epilepsy \\

\bottomrule
\end{tabular}
\end{table}


\bibliographystyle{plain}
\bibliography{references}

\end{document}
