\documentclass{article}

\usepackage[utf8]{inputenc}

\usepackage{booktabs} 

\usepackage{array}

\begin{document}

	\section*{Bibliographical references}
			%intro and foundational works to have as introduction.test papers.
			Recurrence Quantification Analysis (RQA) and Cross-Recurrence Quantification Analysis (CRQA) are
			nonlinear methods for the analysis of nonstationary time series; such as EEG signals. 
			The offer quantification of the recurring patterns in phase space trajectories \cite{trulla1996, webber2005}. 
			Introduced by Trulla et al.\cite{trulla1996} and expanded by Webber and Zbilut\cite{webber2005}, RQA measures metrics 
			like recurrence rate, determinism, and laminarity to capture dynamic system behavior. 
			Thomasson et al.\cite{thomasson2002} in their work, demonstrated RQA’s applicability to EEG, highlighting its robustness to noise
			and nonstationarity. Marwan et al.\cite{marwan2013} further advanced recurrence plot techniques,
			emphasizing their utility in detecting dynamic transitions in neuroimaging. 
			These foundational works underpin the application of RQA and CRQA to EEG studies in epilepsy, 
			cognitive disorders, and beyond, as explored in this review.

	

			%ok, eeg
			Frolov et al.\cite{frolov} (2020) proposed an approach to analyze frequency based multiplex brain networks
			using recurrence quantification analysis (RQA) 
			on EEG data, where they have demonstrated how recurrence-based 
			synchronization indices can effectively capture 
			both within-frequency (intralayer) and cross-frequency (interlayer) 
			functional connectivity during cognitive tasks. 
			Their work showed that RQA is particularly suitable for analyzing 
			non-stationary EEG signals and revealed
			important insights about the evolution of functional connectivity 
			patterns during prolonged cognitive tasks. In addition the dataset
			used in this research are openly available in a Figshare repository.

			%schizophrenia, fmri, ok
			Lombardi et al.\cite{Lombardi2014} have investigated the 
			nonlinear properties in fMRI BOLD signals 
			during a working memory task in 
			schizophrenic patients and healthy controls. 
			They have attempted by using RQA, to analyze recurrence plots 
			for quantifying determinism, trapping time, 
			and maximal vertical line length 
			in functionally relevant brain clusters. 
			Outcome revealed differences in 
			the dynamics between the two groups, 
			and more specific in working memory and DMN areas. 
			While their work have focused on fMRI, the methodology can be adapted also into
			EEG signals, which can offer a higher resolution for capturing rapid neural dynamics.

			%ok. schizophrenia, fMRI
			Kang et al. (2023)\cite{kang}, in their study explore the dynamics and functional connectivity of the 
			Default Mode Network (DMN) in schizophrenia, applying RQA-CRQA on resting-state fMRI data. 
			Findings include decreased \textit{determinism} between specific DMN regions 
			(vMPFC-posterios cingulate and vMPFC-precuneus) in first-episode schizophrenia patients, 
			as a signal of disturbed predictability of functional interactions. 
			Moreover, their results achieve to correctly classify using SVM(support vector machine)
			schizophrenia patients from healthy controls with 77\% classification accuracy.

			%ok.   eeg,Alzheimer
			Núñez et al. \cite{nunez2020characterization} in their work, have analyzed
			resting-state EEG recordings from subjects with mild cognitive impairment(MCI), 
			Alzheimer's disease(AD), and healthy ground truth controls in order to detect 
			frequency based changes into their brain dynamics. 
			By blending wavelet based Kullback–Leibler divergence
			(KLD) for capturing non-stationarity,
			and two recurrence quantification analysis (RQA)
			metrics(\textit{entropy of the recurrence point density}
			and the \textit{median of the recurrence point density}) insights have been
			extracted related to neurodegeneration presence.
			Research's findings show that MCI and AD are presenting notable changes in 
			the recurrence structure and non-stationarity of EEG signals,
			and more specifics on the theta and beta frequency bands.
			Therefore, recurrence based dynamics show a capability as potential 
			biomarkers for monitoring and detecting early Alzheimer's disease and its progression.

			%epilepsy-ok
			Fan and Chou \cite{fan2019detecting} have also proposed 
			an approach for real-time epileptic seizure detection
			using as a method the analysis of temporal synchronization 
			patterns of EEG signals with recurrence networks and spectral graph theory. 
			Recurrence plots are used for the modeling of the EEG dynamics, 
			extracting graph theory's features for quantifying the synchronization. 
			Results showed high sensitivity of 98.48\% and low latency
			(6 seconds) for detecting seizure on the CHB-MIT dataset, 
			performing better than other RQA measures.  

			% Alzheimer, fmri, ok
			Researchers in \cite{rezaei},
			have applied RQA on resting-state fMRI data from 
			TgF344-AD rats(a transgenic rat model which will eventually develop Alzheimer’s disease)
			and their healthy-control counterparts wild-type rats(WT),
			in order to detect early stage biomarkers for the disease.
			By analyzing Default Mode-Like Network (DMLN) 
			using RQA metrics(\textit{entropy, recurrence rate, determinism 
			and average diagonal line length}) 
			changes have been detected in regions of 
			the basal forebrain, hippocampal fields (CA1, CA3), and visual 
			cortices (V1, V2). Also on the study's findings include reduced predictability in 
			WT rats with aging, while AD rats exhibited less decline
			in predictability, suggesting some unknown yet countereacting mechanisms. 
			This study highlights RQA's sensitivity for nonlinear dynamics 
			in preclinical AD and the code used is also publicly available.


			% EEG, aging aisthitiriako-kinitiko systima. ok.
			Author in \cite{pitsik}(2025), investigated changes related to aging in 
			brain sensorimotor systems using 
			RQA and theta-band functional connectivity in EEG signals. 
			In the study a VR experimental paradigm was 
			utilized with auditory stimulus across different age groups(young and elder subjects). 
			Key findings revealed that elder subjects present 
			decreased EEG complexity during motor preparation stages as 
			measured by RQA metrics (\textit{$\Delta$RR and $\Delta$RTE}), 
			and had increased theta band functional connectivity 
			highlighting the potential of RQA in detecting 
			age related biomarkers that were not detectable using 
			standalone signal spectral analysis.

			%cognitive, eeg OK.
			Guglielmo et al. (2022)\cite{guglielmo}  
			utilized RQA features extracted 
			by EEG signals for the purpose of classification
			of cognitive performance during mental arithmetic tasks. 
			They used frontal and parietal EEG signals 
			and analyzed them, from 36 participants by extracting 
			six RQA metrics (\textit{recurrence rate, determinism, 
			laminarity, entropy, maximum diagonal line length and average diagonal line length}) 
			from four electrodes (F7, Pz, P4, Fp1). 
			Afterwards they applied machine learning(ML) classifiers 
			(SVM, Random Forest, and Gradient Boosting) and
			they reached accuracy of classification above 0.85, 
			showing the potential that RQA holds for 
			generalizing on nonlinear dynamics.
			
			%% epilepsy, CRQA half-ok
			Yang and co-authors \cite{yang2019dynamical}, employed stereo-electroencephalography (sEEG) 
			recordings from 10 patients with refractory focal epilepsy to analyze dynamical differences 
			among discreet epileptic states (inter-ictal, pre-ictal, and ictal) and regions. 
			Using recurrence plots and CRQA, they have identified epileptogenic channels with longer diagonal
			structures in RPs, which is a sign of more deterministic and recurrent dynamics. 
			Their findings point out that synchronization between epileptogenic channels strengthened 
			while seizures events occur, suggesting these regions dominate the epileptic network's dynamics.


			%epilepsy, sEEG , RQA ok
			Lopes et al. (2020)\cite{lopes} have proposed a 
			combinatorial framework 
			by mixing RQA with dynamic functional network (dFN) analysis,
			applying it to both MEG and stereo EEG data. 
			The methodology they described is split 
			into five steps: data segmentation, 
			functional network inference, distance computation alongside networks, 
			recurrence plot construction and finally RQA. 
			The study demonstrated that functional networks in epilepsy 
			patients recur more quickly than in healthy controls, suggesting RQA on
			dFNs could serve as a potential biomarker.
			For the EEG dataset investigation, they have showed that the pre-ictal 
			networks shown higher recurrence rates 
			than post-ictal periods, with the $\tau$-recurrence rate ($RR_{\tau}$) proving particularly 
			effective for seizure detection.
			
			%eeg, epilepsy, RQA, OK
			Rangaprakash~\cite{rangaprakash2014} have proposed an application of RQA for the study of
			brain connectivity using multichannel EEG signals. In its work,
			a new CRQA-based feature was proposed (Correlation between 
			Probabilities of Recurrence (CPR)), a nonlinear and non-parametric 
			phase synchronization technique. Afterwards it was utilized for the analysis 
			of functional connectivity in epilepsy subjects during eyes-open/eyes-closed conditions.
			The results demonstrated that CPR outperformed other known traditional 
			linear methods on distinguishing seizure and pre-seizure states, 
			identifying epileptic foci, and differentiating alongside eyes-open and eyes-closed conditions. 

			%npsle
			In their research, Pentari et al.\cite{pentari22} have applied CRQA to resting-state fMRI data 
			for examining the dynamic functional connectivity on patients with neuropsychiatric systemic 
			lupus erythematosus (NPSLE). Results contain the fact that CRQA metrics, such as determinism,
			appear more sensitive than conventional static functional connectivity methods in order to
			identify aberrant connectivity patterns that correlated with visuomotor performance. 
			The study focused on 16 frontoparietal regions and found that CRQA could detect 
			both increased and decreased connectivity in NPSLE patients compared against the healthy controls. 
			Building on these findings, Pentari et al.\cite{pentari23} subsequently expanded 
			the investigation to whole brain network analysis in a larger cohort. 
			In this study they demonstrate the capability of CRQA to integrate multiple recurrence metrics 
			for revealing both hyperconnectivity in parietal regions (angular gyrus and superior parietal lobule) 
			and hypoconnectivity in medial temporal structures (hippocampus and amygdala). 
			Notably, the dynamic connectivity measures showed stronger associations with cognitive 
			performance than structural measures, particularly for verbal episodic memory. 

			%eeg-epilepsy
			Recent studies have demonstrated the effectiveness of RQA in 
			analyzing EEG signals for epilepsy detection. 
			Gruszczyńska et al.\cite{gruszczynska2019} have applied RQA
			in order to distinguish epileptic from healthy patients using EEG recordings 
			from frontal and temporal lobe electrodes (Fp1, Fp2, T3, T4). 
			In their findings they have showed that the epileptic signals present more periodic
			dynamics in comparison to healthy controls, by as evidenced by higher values of 
			RQA parameters such as determinism,
			laminarity, and longest diagonal line. The study combined RQA with
			Principal Component Analysis for dimensionality reduction and visualization, achieving 86.8\% 
			classification accuracy with SVM. This work is particularly relevant as it demonstrates RQA's capability
			to identify pathological patterns in resting-state EEG without requiring seizure events during recording.

			Another study utilizing advanced nonlinear analysis techniques for neural correlation investigation to
			cognitive functions \cite{mo} used \textit{stereoelectroencephalography (sEEG)} combined alongside RQA 
			for the examination of the relationship of the DMN and empathy. 
			Correlations have been detected relating specific RQA metrics 
			(mean diagonal line length, entropy of diagonal line lengths, trapping time) 
			and empathy scores, particularly within DMN subsystems. 

			%epilepsy
			Regarding epilepsy diagnosis, authors in \cite{palanisamy2024} proposed a new framework 
			utilizing the combintation of RQA with genetic algorithms and Bayesian classifiers for 
			identifying corresponding biomarkers for seizure detection. 
			They utilized five distance norms (e.g., Euclidean, Mahalanobis) and multiple thresholds 
			for extracting recurrence features from EEG signals, achieving 100\% classification accuracy. 
			More specific, the \textit{transitivity} feature has shown capability of a highly discriminative biomarker, 
			performing better compared to traditional linear methods. 

			%epilepsy
			Ngamga et al.\cite{ngamga2016} studied the performance achieved of RQA and Recurrence Network (RN) measures in identifying 
			pre-seizure states from multi-day, multi-channel intracranial EEG (iEEG) 
			recordings of epilepsy patients. 
			Results highlighted the correlation among RQA measures (determinism, laminarity, and mean recurrence time) in 
			detecting seizure precursors, while RN measures (average shortest path length and network transitivity) provided 
			complementary but not so consistent insights than using the application of RQA measures alone.

			%% modeling, RQA, ok
			In addition there have been works where simulated data 
			have been used in conjunction with RQA.
			Lameu et al.\cite{lameu2018}, investigated burst phase synchronization in neural networks using RQA. 
			They analyzed two network types; a small-world network and a network of networks 
			(to mimic better the real human brain), using coupled Rulkov maps to model bursting neurons. 
			By applying RQA, they identified synchronized neuron groups and quantified their 
			sizes during synchronization transitions. The study showed that RQA measures 
			(\textit{recurrence rate, laminarity inspired}(custom feature)\textit{, and average structure size}) complement 
			traditional order parameters by revealing localized synchronization patterns, 
			such as the formation and growth of synchronized clusters.
			
			%ok, ASD
			Heunis and co-authors\cite{heunis2018} have utilized resting state EEG and RQA in order to
			distinguish individuals of ages 0-18 of two categories; ASD(autism spectrum disorder) and typically developing.
			They have extracted RQA features and tested various linear and nonlinear classifiers achieving 92.9\% classification
			accuracy with nonlinear SVM classifier.



		\begin{table}[h]
		\centering
		\caption{Comparison among the retrieved studies using recurrence analysis}
		\label{tab:comparison}
		\begin{tabular}{@{}lcccc@{}}
		\toprule
		\# & Reference & Modality & Analysis Methods & Network Type \\
		\midrule

		1  & Frolov et al. (2020) & EEG & RQA, CRQA & Multiplex functional networks \\
		2  & Kang el al. (2023) & fMRI & RQA, CRQA & DMN, schizophrenia \\
		3  & Rezaei el al. (2023) & fMRI & RQA & Default model-like network, AD \\
		4  & Lameu et al. (2018) & --- & RQA & Small-world \& cluster network \\
		5  & Lombardi et al. (2014) & fMRI & RQA & schizophrenia,working memory \\
		6  & Pitsik E. (2025) & EEG & RQA & aging \\
		7  & Guglielmo et al. (2022) & EEG & RQA & cognitive tasks \\
		8  & Lopes et al. (2020) & sEEG, MEG & RQA & epilepsy \\
		9  & Pentari et al. (2022) & fMRI & RQA, CRQA & NPSLE \\
		10 & Pentari et al. (2023) & fMRI & CRQA & NPSLE  \\
		11 & Gruszczyńska et al. (2019) & EEG & RQA & epilepsy \\
		12 & Mo et al. (2022) & sEEG & RQA & DMN, epilepsy \\
		13 & Palanisamy et al. (2024) & EEG & RQA & epilepsy \\
		14 & Ngamga et al. (2016) & EEG & RQA,RN & epilepsy \\
		15 & Fan and Chou (2019) & EEG & RQA,RN & epilepsy, seizure detection \\
		16 & Nunez et al. (2020) & EEG & RQA & AD \\
		17 & Yang et al. (2019) & sEEG & RQA,CRQA & epilepsy \\
		18 & Rangaprakash (2014) & EEG & CPR(CRQA-based) & epilepsy \\
		19 & Heunis et al. (2018) & rsEEG & RQA & autism spectrum disorder \\

		\bottomrule
		\end{tabular}
		\end{table}




	\newpage



\bibliographystyle{plain}
\bibliography{references}

\end{document}
