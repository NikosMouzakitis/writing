\documentclass{article}

\usepackage[utf8]{inputenc}

\usepackage{booktabs} 

\usepackage{array}

\begin{document}

	\section*{Bibliographical references}

			%ok.
			Frolov et al.\cite{frolov} (2020) proposed an approach to analyze frequency based multiplex brain networks
			using recurrence quantification analysis (RQA) 
			on EEG data. The authors demonstrated how recurrence-based 
			synchronization indices can effectively capture 
			both within-frequency (intralayer) and cross-frequency (interlayer) 
			functional connectivity during cognitive tasks. 
			Their work showed that RQA is particularly suitable for analyzing 
			non-stationary EEG signals and revealed
			important insights about the evolution of functional connectivity 
			patterns during prolonged cognitive tasks. In addition the dataset
			used in this research are openly available in a Figshare repository.

			%ok. schizophrenia, fMRI
			Kang et al. (2023)\cite{kang}, in their study explore the dynamics and functional connectivity of the 
			Default Mode Network (DMN) in schizophrenia, applying RQA-CRQA on resting-state fMRI data. 
			Findings include decreased \textit{determinism} between specific DMN regions 
			(vMPFC-posterios cingulate and vMPFC-precuneus) in first-episode schizophrenia patients, 
			as a signal of disturbed predictability of functional interactions. 
			Moreover, their results achieve to correctly classify using SVM(support vector machine)
			schizophrenia patients from healthy controls with 77\% classification accuracy.

			%ok.   eeg,AD--
			Núñez et al. \cite{nunez2020characterization} in their work, have analyzed
			resting-state EEG recordings from subjects with mild cognitive impairment(MCI), 
			Alzheimer's disease(AD), and healthy ground truth controls in order to detect 
			frequency based changes into their brain dynamics. 
			By blending wavelet based Kullback–Leibler divergence
			(KLD) for capturing non-stationarity,
			and two recurrence quantification analysis (RQA)
			metrics(\textit{entropy of the recurrence point density}
			and the \textit{median of the recurrence point density}) insights have been
			extracted related to neurodegeneration presence.
			Research's findings show that MCI and AD are presenting notable changes in 
			the recurrence structure and non-stationarity of EEG signals,
			and more specifics on the theta and beta frequency bands.
			Therefore, recurrence based dynamics show a capability as potential 
			biomarkers for monitoring and detecting early Alzheimer's disease and its progression.

			%epilepsy-ok
			Fan and Chou \cite{fan2019detecting} have also proposed 
			an approach for real-time epileptic seizure detection
			using as a method the analysis of temporal synchronization 
			patterns of EEG signals with recurrence networks and spectral graph theory. 
			Recurrence plots are used for the modeling of the EEG dynamics, 
			extracting graph theory's features for quantifying the synchronization. 
			Results showed high sensitivity of 98.48\% and low latency
			(~6 seconds) for detecting seizure on the CHB-MIT dataset, 
			performing better than other RQA measures.  
			


			Researchers in \cite{rezaei},
			have applied RQA on resting-state fMRI data from TgF344-AD rats in order 
			to detect early stage of Alzheimer's disease biomarkers. 
			Analysis has been conducted on the Default Mode-Like Network (DMLN) 
			using RQA metrics(entropy, recurrence rate, determinism and average diagonal line length) 
			and revealed significant changes in regions like 
			the basal forebrain (BFB), hippocampal fields (CA1, CA3), and visual 
			cortices (V1, V2). 
			On the study's findings are included reduced predictability in 
			wild-type (WT) rats with aging, while AD rats exhibited less decline
			in predictability, suggesting compensatory mechanisms. 
			The study highlights RQA's sensitivity to nonlinear dynamics 
			in preclinical AD, offering potential for early diagnosis. Also the code of the
			research is publicly available.

			Lameu et al.\cite{lameu2018} investigated burst phase synchronization in neural networks using 
			RQA. The authors employed coupled Rulkov maps to model bursting neurons in both single small-world networks
			and clustered network-of-networks architectures. Their spatial RQA approach successfully identified 
			synchronized neuron groups and quantified their sizes during synchronization transitions. The study
			demonstrated that RQA measures (recurrence rate, laminarity, and structure size) provide complementary 
			information to traditional order parameters, particularly for detecting localized synchronization patterns. 
			This work is significant for EEG analysis as it shows RQA's capability to detect phase synchronization 
			in complex networks - a key feature in functional brain connectivity studies.

			%schizophrenia
			Lombardi et al.\cite{Lombardi2014} investigate the nonlinear properties of fMRI BOLD signals 
			during a working memory task in schizophrenic patients and healthy controls. 
			Using RQA, analysis has been performed on the recurrence plots for the quantification of 
			determinism, trapping time, and maximal vertical line length 
			in functionally relevant brain clusters. 
			Outcome revealed differences in nonlinear dynamics among the groups, 
			and more specific in working memory and default mode network areas. 
			This study highlights the potential of RQA for discriminating pathological brain states and understanding 
			functional connectivity in complex systems. While their work focused on fMRI, the methodology is adaptable
			to EEG, which offers higher temporal resolution for capturing rapid neural dynamics.

			%?
			Author in\cite{pitsik} (2025), investigated changes related to aging in brain sensorimotor systems using 
			RQA and theta-band functional connectivity in EEG signals. 
			In the study a VR experimental paradigm was utilized with auditory 
			stimuli across different age groups. 
			Key findings revealed that elderly subjects showed decreased EEG complexity during motor preparation stages as 
			measured by RQA metrics ($\Delta$RR and $\Delta$RTE), 
			and had increased theta-band functional connectivity highlighting the potential of RQA in detecting 
			age-related biomarkers that were not evident in conventional spectral analysis. 

			%cognitive
			Guglielmo et al. (2022)\cite{guglielmo} demonstrated the capability of 
			RQA features which were extracted by EEG signals for the purpose of classification
			of cognitive performance during mental arithmetic tasks. Frontal and parietal EEG signals 
			have been analyzed from 36 participants by extracting six RQA metrics (recurrence rate, determinism, 
			laminarity, entropy, maximum diagonal line length and average diagonal line length) from four electrodes (F7, Pz, P4, Fp1). 
			By further application of machine learning(ML) classifiers such as SVM, 
			Random Forest, and Gradient Boosting, researchers
			they reached accuracy of classification above 0.85, 
			showing the potential that RQA hold for generalizing on nonlinear dynamics.

			%rewrite kala.  %epilepsy
			Lopes et al. (2020)\cite{lopes} have proposed a combinatorial framework 
			mixing RQA with dynamic functional network (dFN) analysis,
			applying it to both MEG and stereo EEG data. 
			The methodology they described is split into five steps: data segmentation, 
			functional network inference, distance computation alongside networks, 
			recurrence plot construction and finally RQA. 
			The study demonstrated that functional networks in epilepsy 
			patients recur more quickly than in healthy controls, suggesting RQA of 
			dFNs could serve as a potential biomarker.
			For EEG applications, they showed that pre-ictal networks exhibit higher recurrence rates 
			than post-ictal periods, with the $\tau$-recurrence rate ($RR_{\tau}$) proving particularly 
			effective for seizure detection.

			%npsle
			In their research, Pentari et al.\cite{pentari22} have applied CRQA to resting-state fMRI data 
			for examining the dynamic functional connectivity on patients with neuropsychiatric systemic 
			lupus erythematosus (NPSLE). Results contain the fact that CRQA metrics, such as determinism,
			appear more sensitive than conventional static functional connectivity methods in order to
			identify aberrant connectivity patterns that correlated with visuomotor performance. 
			The study focused on 16 frontoparietal regions and found that CRQA could detect 
			both increased and decreased connectivity in NPSLE patients compared against the healthy controls. 
			Building on these findings, Pentari et al.\cite{pentari23} subsequently expanded 
			the investigation to whole brain network analysis in a larger cohort. 
			In this study they demonstrate the capability of CRQA to integrate multiple recurrence metrics 
			for revealing both hyperconnectivity in parietal regions (angular gyrus and superior parietal lobule) 
			and hypoconnectivity in medial temporal structures (hippocampus and amygdala). 
			Notably, the dynamic connectivity measures showed stronger associations with cognitive 
			performance than structural measures, particularly for verbal episodic memory. 

			%eeg-epilepsy
			Recent studies have demonstrated the effectiveness of RQA in 
			analyzing EEG signals for epilepsy detection. 
			Gruszczyńska et al.\cite{gruszczynska2019} have applied RQA
			in order to distinguish epileptic from healthy patients using EEG recordings 
			from frontal and temporal lobe electrodes (Fp1, Fp2, T3, T4). 
			In their findings they have showed that the epileptic signals present more periodic
			dynamics in comparison to healthy controls, by as evidenced by higher values of 
			RQA parameters such as determinism,
			laminarity, and longest diagonal line. The study combined RQA with
			Principal Component Analysis for dimensionality reduction and visualization, achieving 86.8\% 
			classification accuracy with SVM. This work is particularly relevant as it demonstrates RQA's capability
			to identify pathological patterns in resting-state EEG without requiring seizure events during recording.

			Another study utilizing advanced nonlinear analysis techniques for neural correlation investigation to
			cognitive functions \cite{mo} used \textit{stereoelectroencephalography (sEEG)} combined alongside RQA 
			for the examination of the relationship of the DMN and empathy. 
			Correlations have been detected relating specific RQA metrics 
			(mean diagonal line length, entropy of diagonal line lengths, trapping time) 
			and empathy scores, particularly within DMN subsystems. 

			%epilepsy
			Regarding epilepsy diagnosis, authors in \cite{palanisamy2024} proposed a new framework 
			utilizing the combintation of RQA with genetic algorithms and Bayesian classifiers for 
			identifying corresponding biomarkers for seizure detection. 
			They utilized five distance norms (e.g., Euclidean, Mahalanobis) and multiple thresholds 
			for extracting recurrence features from EEG signals, achieving 100\% classification accuracy. 
			More specific, the \textit{transitivity} feature has shown capability of a highly discriminative biomarker, 
			performing better compared to traditional linear methods. 

			%epilepsy
			Ngamga et al.\cite{ngamga2016} studied the performance achieved of RQA and Recurrence Network (RN) measures in identifying 
			pre-seizure states from multi-day, multi-channel intracranial EEG (iEEG) 
			recordings of epilepsy patients. 
			Results highlighted the correlation among RQA measures (determinism, laminarity, and mean recurrence time) in 
			detecting seizure precursors, while RN measures (average shortest path length and network transitivity) provided 
			complementary but not so consistent insights than using the application of RQA measures alone.




		\begin{table}[h]
		\centering
		\caption{Comparison among the retrieved studies using recurrence analysis}
		\label{tab:comparison}
		\begin{tabular}{@{}lcccc@{}}
		\toprule
		\# & Reference & Modality & Analysis Methods & Network Type \\
		\midrule

		1  & Frolov et al. (2020) & EEG & RQA, CRQA & Multiplex functional networks \\
		2  & Kang el al. (2023) & fMRI & RQA, CRQA & DMN, schizophrenia \\
		3  & Rezaei el al. (2023) & fMRI & RQA & Default model-like network \\
		4  & Lameu et al. (2018) & --- & RQA & Small-world \& clustered networks \\
		5  & Lombardi et al. (2014) & fMRI & RQA & schizophrenia,working memory \\
		6  & Pitsik E. (2025) & EEG & RQA & aging \\
		7  & Guglielmo et al. (2022) & EEG & RQA & cognitive tasks \\
		8  & Lopes et al. (2020) & sEEG, MEG & RQA & epilepsy \\
		9  & Pentari et al. (2022) & fMRI & RQA, CRQA & NPSLE \\
		10 & Pentari et al. (2023) & fMRI & CRQA & NPSLE  \\
		11 & Gruszczyńska et al. (2019) & EEG & RQA & epilepsy \\
		12 & Mo et al. (2022) & sEEG & RQA & DMN, epilepsy \\
		13 & Palanisamy et al. (2024) & EEG & RQA & epilepsy \\
		14 & Ngamga et al. (2016) & EEG & RQA,RN & epilepsy \\
		15 & Fan and Chou (2019) & EEG & RQA,RN & epilepsy, seizure detection \\
		16 & Nunez et al. (2020) & EEG & RQA & Alzheimer \\

		\bottomrule
		\end{tabular}
		\end{table}




	\newpage



\bibliographystyle{plain}
\bibliography{references}

\end{document}
